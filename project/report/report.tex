\documentclass{article}
\usepackage{cite}
\usepackage{url}

\title{Implementazione di un algoritmo Q-Learning con Q discreta e Double Dueling Deep Q-Learning}

\date{2019-06-01}
\author{Michele Proverbio}

\begin{document}
	\pagenumbering{gobble}
	\maketitle
	\newpage
	\pagenumbering{arabic}

	\section{Introduzione}
		Il Reinforcement Learning è un'area del Machine Learning che studia l'apprendimento automatico di un task da parte di un agente. Quest'ultimo è immerso in un ambiente che restituisce all'agente un feedback proporzionale all'efficacia delle sue azioni rispetto al task.

		Lo scopo del progetto è implementare due algoritmi di Q-learning: una implementazione naive con Q discreta, e un algoritmo Double Dueling Deep Q-learning \cite{1509.06461} \cite{1511.06581} sviluppato da DeepMind \cite{DeepMind} per il framework di benchmark Atari \cite{Atari}. Entrambi gli algoritmi verranno testati in ambienti virtuali di OpenAI Gym \cite{open.ai}.

	\section{Q-learning}
	\subsection{}

	\newpage
	\bibliography{mybib}{}
	\bibliographystyle{plain}
\end{document}